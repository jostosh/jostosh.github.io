\documentclass[10pt]{article}
\begin{document}
Hello everyone, I am Siebert and
today I am going to give a talk about Robobrain. A large scale knowledge engine for robots. Robobrain is proposed by Saxena et al. in 2014. Perception, natural language and planning are the fields in robotics, where maybe most research has been. But a lot of research has been done on the seperated fields. In perception, Microsoft has lately broke the state of the art in recognition of images by scoring better on a dataset then a human. Some research has combined two fields together, like Robin Koezen told a week ago about using the Google image recognition for grasping.  The author of this paper was suprised that there was not a research going on, that tried to combine the multiple modules, so they started with Robobrain. The definition the authors give for Robobrain: "Large-scale knowledge engine for robotics". The important word is here "knowledge engine". What is a knowledge engine ? I found the answer by the one of the most famous knowledge base "wikipedia". A knowlege based engine is a computer program that reasons and use a knowledge base to solve complex systems. The three most famous examples of a knowledge base are IBM Watson, that won Jeopardy a couple of years ago, wikipedia and the google knowledge graph, what represent the data for the google search engine. The knowledge base of Robobrain is represented by a directed graph. Who does not know what a directed graph is ?
\textit{Draw a directed graph if they do not know}. 
The nodes consist out of the concepts of the model. This can be anything because there are no constraint on the type of the data.  The edges consist out of the two nodes, and a edge type which needs to be in the edge set. Every node and edge also has a belief.\\
\textbf{belief}\\
That is the probability over the accuracy of the information of a node or edge. \\\\
Coarse feedback: these are binary feedback where a user can "Approve" or "dissaprove" a concept in RoboBrain through it online web interface.\\
Graph feedback: these feedback are elicited on Robobrain graph visualizer, where a user modifies the graph by adding/deleting nodes or edges. 
Robot feedback: Physical feedback given by users directly on the robot (mostly by a button).\\\\
\textbf{Architecture}\\
knowledge acquistion layer is the layer between Robobrain and the different sources of the different data sources. Robobrain primarily collects knowledge through its partner projects and by crawling knowledge bases.\\
Knowledge parser layer  processes the data that is received by the previous layer and convert it to a consistent data type format for the storage layer.\\\\
knowledge storage layer is responsible for storing different representations of the data.  There are three storage modules. On the left side is for the storage of images, point cloud etc. The middle storage is seen as the single source of truth. In the middle storage the important information is send to the next layer and from there it is send to the graph storage, what we saw in the previous slides. \\\\
Knowledge inference layer constains the key processing and machine learning components of Robobrain. I could not find the machine learning algorithm that they were using.\\\\
\textbf{Robot query Library}\\

\end{document}