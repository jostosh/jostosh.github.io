\documentclass[a4paper,10pt]{article}
\usepackage[
        a4paper,% other options: a3paper, a5paper, etc
        left=3cm,
        right=3cm,
        top=3cm,
        bottom=4cm,
        % use vmargin=2cm to make vertical margins equal to 2cm.
        % us  hmargin=3cm to make horizontal margins equal to 3cm.
        % use margin=3cm to make all margins  equal to 3cm.
]{geometry}
\usepackage{amssymb}
\usepackage{amsmath}
\usepackage{amsfonts}
\usepackage{tikz}
\usepackage{listings}
\usepackage[scaled=.85]{sourcecodepro}
\usepackage{color}
\usepackage{comment}
\usepackage{bbold}

\definecolor{codegreen}{rgb}{0,0.6,0}
\definecolor{codegray}{rgb}{0.5,0.5,0.5}
\definecolor{codepurple}{rgb}{0.58,0,0.82}
\definecolor{backcolour}{rgb}{0.95,0.95,0.92}


\lstdefinestyle{mystyle}{
    backgroundcolor=\color{backcolour},   
    commentstyle=\color{codegreen},
    keywordstyle=\color{blue},
    numberstyle=\tiny\color{codegray},
    stringstyle=\color{codepurple},
    basicstyle=\small\ttfamily,
    breakatwhitespace=false,         
    breaklines=true,                 
    captionpos=b,                    
    keepspaces=true,                 
    numbers=left,                    
    numbersep=5pt,                  
    showspaces=false,                
    showstringspaces=false,
    showtabs=false,                  
    tabsize=2
}

\lstset{style=mystyle}
\title{Forehead Idiot\\ {\Large Multi Agent Systems}}
\author{Jos van de Wolfshaar (s2098407)\\
Siebert Looije\\
Diederik Greveling} 
\date{\today}

\begin{document}
\maketitle
\section{General Kripke structure}
\subsection{Kripke model}
For modeling Forehead idiot we consider the structures $M = \langle S, \pi, R \rangle$. We consider the games for 3 players. The states are given by $\boldsymbol s = (s_1,s_2,\ldots,s_m) \in S$, where $s_i \in D = \{ 2,3,4,5,6,7,8,9,10,j,q,k,a \}$. This means that if $s_3 = 5$, then player 3 has 5 on his forehead. Note that $D$ is an ordered set:
$2<3<4<5<6<7<8<9<10<j<q<k<a$.

Since any player can only see those cards that are on the forehead of the other players, the relations are defined as $\boldsymbol s R_i \boldsymbol t$ such that $s_j = t_j,~\forall j \neq i$. A player's card is unique: $s_1 \neq s_2 \neq \cdots \neq s_m$. Note that relations are symmetric and reflexive.

Suppose we take a subset of the total deck $D'\subseteq D$ with $|D'|=n$. There are $n!$ different sequences of cards. If the number of players is $m\leq n$, then for each possible set of cards there will be $(n-m)!$ games with the same sequence of cards for the first $m$ cards. Hence, the total number of possible games $g$ is:
\begin{equation}
 g=\frac{n!}{(n-m)!}
\end{equation}

The propositions in our model will be given by $\boldsymbol P = \{ p_ic_2,p_ic_3,\ldots,p_ic_a,w_i,l_i\}_{i=1}^m$. For example, $p_2 c_7$ means that player 2 holds card 7. The proposition $w_i$ means that player $i$ wins the game, whereas $l_i$ means that player $i$ loses the game.

The valuation will be as follows:
\begin{itemize}
 \item We have that $\pi(\boldsymbol s)(p_ic_j) = \mathbb{t}$ iff $s_i = j$, so $\pi(\boldsymbol s)(p_ic_j) = \mathbb{f}$ iff $s_i \neq j$. 
 \item Furthermore, we have $\pi(\boldsymbol s)(w_i) = \mathbb{t}$ and $\pi(\boldsymbol s)(l_i) = \mathbb{f}$ iff $s_i > s_j$ for all $j \neq i$. In other words, $\pi(\boldsymbol s)(w_i)=\mathbb{f}$ and $\pi(\boldsymbol s)(l_i) = \mathbb{t}$ if for some $j\neq i$ we have that $s_i < s_j$. 
\end{itemize}

\subsection{Theory}
Before the cards are put on the foreheads, every player knows that every player has exactly one card. This can be captured by
\begin{align*}
 K_i \bigwedge_{j=1}^m \bigg((p_j c_2 \wedge \neg p_j c_3 \wedge \cdots \wedge \neg p_j c_a) \vee (\neg p_j c_2 \wedge p_j c_3 \wedge \cdots \wedge \neg p_j c_a) \vee \cdots \vee (\neg p_j c_2 \wedge \neg p_j c_3 \wedge \cdots \wedge p_j c_a) \bigg)
\end{align*}

The moment that the players put the cards on their foreheads it is as if $m$ announcements are made. Every player announces that he knows what the other players' cards are (without actually saying it). This means that every player also knows his remaining possibilities.

In the most general case, we could express this as follows:
\begin{align*}
 M \models \left[ \bigwedge_{i=1}^m K_i \left(\bigwedge_{j\neq i} p_j c_{u^{(j)}} \right) \right] \bigwedge_{i=1}^m \left( K_i \bigvee_{u^{(i)} \notin \{ u^{(j)} \}_{j\neq i}} p_i c_{u^{(i)}} \right)
\end{align*}
where $u^{(j)} \in D'$ is the card that player $j$ is holding. This greatly reduces the number of possible worlds. In fact, if there are $n$ cards and $m$ players, we have that each player has $n-m+1$ alternative worlds. Hence there are $m(n-m+1) - m +1$ possible worlds left. 


\section{Game analysis}
\subsection{$n$ cards game with $m=n$ players}
In the scenario of $n$ cards with $m=n$ players there will be $\displaystyle g = \frac{n!}{(n-m)!} = n!$ possible games. Once the cards are dealt, each player is able to see the other players' cards. Because of the fact that $m=n$ and the fact that all cards are unique, for any pair of states $\boldsymbol s, \boldsymbol t$ such that $\boldsymbol s \neq \boldsymbol t$, we have that at least two cards are changed: $s_i \neq t_i$ and $s_j \neq t_j$ for $i,j\leq m$ and $i\neq j$.

This implies that there are only reflexive relations: $\boldsymbol s R_i \boldsymbol s$, since for any other $\boldsymbol t \neq \boldsymbol s$, we require that all cards but one are the same. For any state $\boldsymbol s$ this relation is only satisfied by $\boldsymbol s$ itself.

Therefore, once a player sees the other players' cards, he will know his own card, since there is only one state to go to. Hence, each player knows whether he wins or loses. Therefore, from a game theoretic point of view, this scenario is not particularly interesting.

TODO: maybe formalize this more...

\subsection{$n$ cards with $m<n$ players}
A more difficult scenario arises when there are $n$ cards and $m < n$ players. In this case there are relations $\boldsymbol s R_i \boldsymbol t$ for those pairs of $\boldsymbol s$ and $\boldsymbol t$ such that $s_j = t_j$ for all $j\neq i$.

\subsubsection{The probability of winning the game}

\paragraph{Estimating one's own odds}
Since there are $n-m$ cards that have not been dealt, each player has $n-m+1$ possibilities for his own card and, consequently, $n-m+1$ possible outcomes of the game. To determine the probability for a player to win, let $T_i \subseteq S$ be the subset of possible states such that $\boldsymbol s R_i \boldsymbol t$. Note that $|T_i| = n-m+1$. Now let $T^{(w_i)}_i \subseteq T_i$ be a subset of states such that for $\boldsymbol t \in T^{(w_i)}_i$ we have that $(M,\boldsymbol t) \models w_i$. The probability that player $i$ wins based on the information available is:
\begin{equation}
\mathcal{P}^{(w_i)}_i = \frac{|T^{(w_i)}_i|}{|T_i|} = \frac{\lvert T^{(w_i)}_i\rvert}{n-m+1}
\label{eq:winprob}
\end{equation}

The relations are symmetric. Hence, $\mathcal{P}^{(w_i)}_i$ will be the same for all states.

\paragraph{Estimating another player's odds}
Let $T^{(w_j)}_i \subseteq T_i$ be the subset of states such that for all states $\boldsymbol t \in T^{(w_j)}_i$ we have that $(M,\boldsymbol t) \models w_j$. Based on this information and perhaps not surprisingly, the probability for player $j$ to win this game from player $i$'s point of view will be given by:
\begin{equation}
\mathcal{P}^{(w_j)}_i = \frac{|T^{(w_j)}_i|}{|T_i|} = \frac{\lvert T^{(w_j)}_i\rvert}{n-m+1} \label{eq:prob}
\end{equation}


\subsubsection{Playing strategies}
\paragraph{Calling a bet}
Before we get to the optimal strategy, let us define the utility. If $\mathcal G^{(k)}_i$ denotes the `gain' of winning the game at round $k$, the utility will be given by:
\begin{equation}
 U^{(w_i)}_i = \mathcal G^{(k)}_i \mathcal P^{(w_i)}_i
\label{eq:util}
 \end{equation}
Let $\mathcal{L}^{(k)}_i$ be the loss of folding or losing the game after calling round $k$. In other words, $\mathcal L^{(k)}_i$ denotes the amount of coins that the player $i$ will have to bet for playing until round $k$. Note that $\displaystyle \mathcal G^{(k)}_i = \sum_{j\neq i} \mathcal L^{(k)}_j$. 

The utility of losing a game will be given by:
\begin{equation}
 U^{(l_i)}_i = -\mathcal L^{k}_i (1-\mathcal P^{(w_i)}_i)
 \label{eq:loss}
\end{equation}

Now we can define a player's strategy for calling any bet as follows. Let $C_i^{(k)} \in \{\mathbb t, \mathbb f\}$ denote the decision of player $i$ at round $k$, where $\mathbb t$ corresponds to calling the bet and $\mathbb f$ corresponds to folding. The decision will be given by:
\begin{equation}
C_i^{(k)}  =
\begin{cases}
 \mathbb t & \text{if } \gamma_i U^{(w_i)}_i + U^{(l_i)}_i \geq 0\\
 \mathbb f & \text{otherwise}
\end{cases}
\label{eq:strat}
\end{equation}
In equation \ref{eq:strat}, $\gamma_i$ will characterize the player's characteristic strategy. If $\gamma_i = 1$, the player has an optimal strategy. If $\gamma_i > 1$ the player exhibits a risky strategy. If $\gamma_i < 1$ the player exhibits a safe strategy.

We can also decide to make the gain $\mathcal G^{(k)}_i$ a bit more specific:
\begin{equation}
\mathcal G^{(k)}_i = {\color{red} \sum_{j=1}^{i-1} \mathcal L^{(k)}_j} + {\color{blue} \sum_{j=i+1}^m \mathcal L^{(k-1)}_j} + {\color{magenta} \sum_{j=i+1}^m \{ C^{(k)}_j\} (\mathcal L^{(k)}_j - \mathcal L^{(k-1)}_j)}
\label{eq:gain}
\end{equation}
where we have the {\color{red} bet of all players hat have already called the current bet}, the {\color{blue} bet of all players that still have to call the current bet} and {\color{magenta} the bets that those players are going to place this round}. In this equation $\{C^{(k)}_j\} = 1$ if $C^{(k)}_j=\mathbb{t}$ and $\{C^{(k)}_j\} = 0$ otherwise.

It is interesting to see if we can let the players deduce the value of $\gamma_i$ for the other players. Until then, it seems reasonable to simplify equation (\ref{eq:gain}) to:
\begin{equation}
\mathcal{G}^{(k)}_i = \sum_{j\neq i} \mathcal L^{(k)}_j
\label{eq:sgain}
\end{equation}

\paragraph{Raising the current bet: bluffing}
A player can also decide to raise the current bet. The main reason for raising your bet is to make the other players decide to quit the game, or to increase the final gain. A player now bluffs if, after its bluffing bet, $\gamma_i U^{(w)}_i + U^{(l)}_i < 0$. 

Based on the information that is available, a player can estimate the utilities of other players. Hence, it can make its own estimations of the other players' gains: $\langle \mathcal G^{(k)}_j \rangle_i$, which is the estimated gain for player $j$ from player $i$'s point of view. 

The estimations can be obtained by using equations (\ref{eq:util}) through (\ref{eq:sgain}), but now we will start with some estimated probability $\mathcal P^{w_j}_i$, from equation (\ref{eq:prob}).

Once we have the estimated gains we can make an estimation of the other player's decisions. For bluffing we want to achieve that $C^{(k)}_j = \mathbb{f}$ for the players $j\neq i$. Hence, if a player decides to bluff, he should try to ensure that $\langle C^{(k)}_j \rangle_i = \mathbb{f}$ for those player's that have a higher probability of winning. Formally, it should make at least those players $j$ fold such that $\mathcal P^{(w_j)}_i > \mathcal P^{(w_i)}_i$. 

We will consider bluffing from another perspective after we have discussed the theory about strategies in section \ref{sec:strat}.

\paragraph{Raising the current bet: increasing the gain}
If a player is confident of winning the round, he can decide to raise the current bet. A player should try to accomplish that $C^{(k)}_j = \mathbb{t}$ while it should also maximize the expected gain. 

\subsection{Reasoning about strategies}
\label{sec:strat}
We now want to address the agents' reasoning during a game. We will start out small by simplifying the rules of the game. We will only have a single betting round. Each player can choose to either call the minimum bet of 1 coin, or fold. The player that is seated on the left side of the dealer has to put in a blind bet of 1 coin. This is to encourage other players for participating (knowing that there is at least something at stake).

Let's extend our set of propositions to $\boldsymbol{P}'=\boldsymbol{P} \cup \{r_i,o_i,h_i\}_{i=1}^m$ in which $r_i$ will resemble a risky strategy for player $i$, $o_i$ will be used for optimal strategy and $h_i$ for safe or `harmless' strategy. A risky strategy will mean that $\gamma_i=1.1$, an optimal strategy will mean that $\gamma_i=1.0$ and a harmless strategy will resemble $\gamma_i=0.9$.

Note that we have:
\begin{align*}
 M &\models K_i r_i \vee K_i o_i \vee K_i h_i\\
 M &\models K_i \bigwedge_{j=1}^m\bigg( (r_j \wedge \neg o_j \wedge \neg h_j ) \vee
					(\neg r_j \wedge o_j \wedge \neg h_j) \vee
					(\neg r_j \wedge \neg o_j \wedge h_j)\bigg)
\end{align*}
for $i=1,2,\ldots,m$.

\subsubsection{First round}
First we can compute the loss for each player of participating. This is simply the amount of coins, so $\forall i,~\mathcal{L}^{(1)}_i=1$. Thus, we can simplify the utility for losing (equation \ref{eq:loss}):
\begin{equation}
U^{(l_i)}_i = -(1 - \mathcal{P}^{(w_i)}_i)
\label{eq:lossutil}
\end{equation}

At the first round, the players know nothing about each other's strategies (given by $\gamma_i$). Hence, they will play based solely on what they see by looking at the opponents' foreheads.

There is no way to estimate the decisions of the other players yet. Hence, we will compute the gain simply by adding the losses of the other people (equation \ref{eq:sgain}). Note that this should be simply $m-1$.

Once we have the gain, we can compute the utility of winning a game and so we can also obtain the decision for each player (equation \ref{eq:strat}).

\subsubsection{Ending the round}
At the end of a round, the players are allowed to see their own cards. Hence they now know on which cards the decision of each opponent was based on. It is as if each player gets to know the utility of all other players during the round that just ended.

First, suppose that $C^{(k)}_i = \mathbb{t}$. Now any player $j\neq i$ knows that the threshold of 0 was exceeded. Hence each player $j$ knows 
\begin{align*}
\gamma_i U^{(w_i)}_i + U^{(l_i)}_i &\geq 0\\
\gamma_i &\geq -\frac{U^{(l_i)}_i }{U^{(w_i)}_i}\\
\end{align*}
So the lower bound of $\gamma_i$ from $j$'s perspective is increased.

Second, suppose that $C^{(k)}_i = \mathbb{f}$. Now any player $j\neq i$ knows:
\begin{align*}
\gamma_i U^{(w_i)}_i + U^{(l_i)}_i &< 0\\
\gamma_i &< -\frac{U^{(l_i)}_i }{U^{(w_i)}_i}\\
\end{align*}

In either case, we could consider it to be an announcement of the possibilities that are left. For example, if players $j\neq i$ find that $\gamma_i < 1.5$, it is as if it is announced that $r_i \vee o_i \vee h_i$. In other words, we might not gain any information. However, a series of rounds could enable agents $j$ to deduce $\gamma_i$. For example, let's say that three rounds are played, in which $\gamma_i < 1.5,\gamma_i \geq 0.93,\gamma_i < 1.06$. This sequence of announcements can be captured by:
\begin{equation*}
M\models [r_i \vee o_i \vee h_i][\neg h_i][\neg r_i] \bigwedge_{j=1}^m K_j o_i
\end{equation*}

\subsubsection{Knowing another one's strategy}
Once a player knows another player's strategy, it can use this to estimate its own odds based on another player's decision. If player $i$ calls, then another player $j$ knows that 
\begin{align*}
\gamma_i U^{(w_i)}_i + U^{(l_i)}_i &\geq 0\\
\gamma_i U^{(w_i)}_i &\geq -U^{(l_i)}_i\\
\gamma_i \mathcal G^{(k)}_i \mathcal P^{(w_i)}_i &\geq 1 - \mathcal{P}_i^{(w_i)} & \text{eq. (\ref{eq:lossutil}) and (\ref{eq:util})}\\
\gamma_i \mathcal G^{(k)}_i \mathcal P^{(w_i)}_i + \mathcal{P}_i^{(w_i)} &\geq 1 \\
\mathcal{P}_i^{(w_i)}  (\gamma_i \mathcal G^{(k)}_i + 1) &\geq 1 \\
\mathcal{P}_i^{(w_i)} &\geq \frac{1}{\gamma_i \mathcal G^{(k)}_i + 1}\\
\frac{\lvert T^{(w_i)}_i\rvert}{n-m+1} &\geq \frac{1}{\gamma_i \mathcal G^{(k)}_i + 1} & \text{eq. (\ref{eq:winprob})}\\
\lvert T^{(w_i)}_i\rvert&\geq \frac{n-m+1}{\gamma_i \mathcal G^{(k)}_i + 1}\\
\lvert T^{(w_i)}_i\rvert&\geq \frac{n-m+1}{\gamma_i (m-1) + 1} & (\mathcal G^{(k)}_i = m-1)
\end{align*}
By similar reasoning, if player $i$ folds, then player $j$ knows that
\begin{align*}
\lvert T^{(w_i)}_i\rvert&<\frac{n-m+1}{\gamma_i (m-1) + 1} 
\end{align*}
For simplicity, we assume that every player expects his gain to be as in equation (\ref{eq:sgain}). Hence, $\mathcal{P}_i^{(w_i)}$ will be the only unknown.

If player $j$ knows such restrictions for $|T^{(w_i)}_i|$, it can go through its possible worlds. For every world, it should expand the possibilities for player $i$. It can then see whether this restriction on $|T^{(w_i)}_i|$ is satisfied by the expansion of that particular world. In essence, we need to revisit the Kripke model in which we have all possible worlds before any cards are drawn. For each $\boldsymbol s \in T_j$ (note the subscript $j$) we have to look at the states $\boldsymbol t$ such that $\boldsymbol s R_i \boldsymbol t$. Let us denote the set $\mathcal T^{\boldsymbol s}_i = \{\boldsymbol t | \boldsymbol t \in S \text{ such that } \boldsymbol s R_i \boldsymbol t \}$. For each $T^{\boldsymbol s}_i$ we count the occurrences of $w_i$. If the number of $w_i$ satisfies the restriction of $|T^{(w_i)}_i|$, then $\boldsymbol s$ is still a valid alternative for $j$. 

The consequence of this in terms of public announcement logic, might be viewed as an announcement about the possibilities for the card that player $j$ is holding. If the possibilities that remain after such an announcement form a proper subset of $T_j$ (note the subscript $j$), then player $j$ can re-estimate its own utilities for winning and losing.

It is difficult to describe a generic statement of such announcements, as the actual announcements depend on the number of players and cards that are in the game.

\subsubsection{Bluffing}
Now that we have established the effect of calling and folding in case of known strategies, we can more accurately analyze the consequence of bluffing. Once player $j$ knows the strategy of $i$, player $i$ knows that player $j$ knows his strategy. Hence, player $i$ could deviate from its strategy and `bluff' to make any player $j\neq i$ believe that their cards are worse than they actually are. It is important to note that bluffing is extremely risky in this game, as a player does not know his own card.

If player $i$ bluffs, he attempts to set the lower bound for $|T^{(w_i)}_i|$ from $j$'s perspective higher than it actually is. A bluff is successful if the subset $T_j$ that remains is such that player $j$ decides to fold.

\section{Example game}
Let us now consider a simple game in which we apply the general theory that was outlined above. We will consider a relatively simple game. There will be 3 players and 8 cards. Before any cards are drawn the number of possible games $g$ is 
\begin{equation}
 g=\frac{8!}{(8-3)!}=336
\end{equation}

\subsection{First round}
In the first round, the first player draws a 6, the second player draws a 2 and the third player draws a 5. Once these cards are drawn, we have $m(n-m+1) - m +1 = 3(8-3+1) - 3 + 1=16$, so each player has 6 alternative games.

From the applet, we can see that the probabilities are $\mathcal{P}^{(w_1)}_1=\frac{2}{3}$, $\mathcal{P}^{(w_2)}_2=\frac{1}{2}$, $\mathcal{P}^{(w_3)}_3 = \frac{1}{2}$. It is interesting to see that player two can estimate its own odds of winning at $\frac{1}{2}$ while it is obvious for the other players that he is going to lose. Let us assume that $r_i$ corresponds to $\gamma_i=1.5$, $o_i$ corresponds to $\gamma_i=1$ and $h_i$ corresponds to $\gamma_i=0.4$. We will have the situation in which $\gamma_1 = 1.5$, $\gamma_2 = 1$ and $\gamma_3=.4$.

We can compute the utilities of winning for each player:
\begin{align*}
 U^{(w_1)}_1 &= \mathcal G^{(1)}_1 \mathcal P^{(w_1)}_1 = 2 \cdot \frac{2}{3} = \frac{4}{3}\\
 U^{(w_2)}_2 &= \mathcal G^{(1)}_2 \mathcal P^{(w_2)}_2 = 2 \cdot \frac{1}{2} = 1\\
 U^{(w_3)}_3 &= \mathcal G^{(1)}_3 \mathcal P^{(w_3)}_3 = 2 \cdot \frac{1}{2} = 1
\end{align*}
And the utilities for losing:
\begin{align*}
 U^{(l_1)}_1 &= -\mathcal L^{(1)}_1 (1-\mathcal P^{(w_1)}_1) = -1\cdot \frac{1}{3} = -\frac{1}{3}\\
 U^{(l_2)}_2 &= -\mathcal L^{(1)}_2 (1-\mathcal P^{(w_2)}_2) = -1\cdot \frac{1}{2} = -\frac{1}{2}\\
 U^{(l_3)}_3 &= -\mathcal L^{(1)}_3 (1-\mathcal P^{(w_3)}_3) = -1 \cdot \frac{1}{2} = -\frac{1}{2}
\end{align*}

We will have the following:
\begin{align*}
\begin{array}{clll}
 \gamma_1 U^{(w_1)}_1 + U^{(l_1)}_1 &= 1.5\cdot \frac{4}{3} - \frac{1}{3} = \frac{5}{3} &\geq 0 &\Rightarrow C^{(1)}_1 = \mathbb{t}\\
 \gamma_2 U^{(w_2)}_2 + U^{(l_2)}_2 &= 1\cdot 1 - \frac{1}{2} = \frac{1}{2} &\geq 0 &\Rightarrow C^{(1)}_2 = \mathbb{t}\\
 \gamma_3 U^{(w_3)}_3 + U^{(l_3)}_3 &= 0.4\cdot 1 - \frac{1}{2} = -0.1 &< 0 &\Rightarrow C^{(1)}_3 = \mathbb{f}
\end{array}
\end{align*}

Now the round ends. Hence, each player knows $\displaystyle -\frac{U^{(l_i)}}{U^{(w_i)}_i}$ for all players:
\begin{align*}
 \gamma_1 &\geq -\frac{U^{(l_1)}_1}{U^{(w_1)}_1} = -\frac{-1/3}{4/3}=\frac{1}{4}\\
 \gamma_2 &\geq -\frac{U^{(l_2)}_2}{U^{(w_2)}_2} = -\frac{-1/2}{1}=\frac{1}{2}\\
 \gamma_3 &< -\frac{U^{(l_3)}_3}{U^{(w_3)}_3} = -\frac{-1/2}{1}=\frac{1}{2}
\end{align*}
Hence, it is as if it is announced that $[h_3]$ and so we have:
\begin{equation}
 M \models [h_3] (K_1 h_3 \wedge K_2 h_3 \wedge K_3 h_3)
\end{equation}
\subsection{Second round}
We will now demonstrate how the other player's can use the strategy of player 3 to re-estimate their own probabilities based on the decision of player 3.

In the second round, player 1 draws 2, player 2 draws 3 and player 3 draws 4. Since all players have relatively low cards, each player will have an optimistic view of the game's outcome. Player 3 is to decide first. It is clear that $\mathcal{P}^{(w_3)}_3=1$. Hence, player 3 decides to call. For player 1 and 2, this means that
\begin{equation*}
 |T_3^{(w_3)}| \geq \frac{n-m+1}{\gamma_3 (m-1)+1} = \frac{8-3+1}{0.4(3-1)+1} = 3\frac{1}{3}
\end{equation*}
Since we know that $|T_3^{(w_3)}|$ should be an integer, we have that $|T_3^{(w_3)}|\geq 4$. Note that there are $n-m+1$ alternatives for each player, so there are $8-3+1=6$ alternatives. 

It is important to note that at this point $\mathcal{P}^{(w_2)}_2 = \mathcal{P}^{(w_3)}_3 = \frac{5}{6}$.

\subsubsection{Updating player 1's view}
At this point we can say that player 1 knows what the cards of player 2 and 3 are: 
\begin{equation*}
M' \models K_1 p_2c_3 \wedge K_1p_3c_4 
\end{equation*}
where $M'$ is the model that remains after the initial announcement of drawing the cards. Furthermore, he knows:
\begin{equation*}
 M' \models K_1(p_1 c_2 \vee p_1 c_5 \vee p_1 c_6 \vee p_1 c_7 \vee p_1 c_8 \vee p_1 c_9)
\end{equation*}
\begin{itemize}
 \item If $p_1 c_2$ were true, then $|T_3^{(w_3)}| = 6$, so this situation satisfies the constraint of $|T_3^{(w_3)}| \geq 4$.
 \item If $p_1 c_5$ were true, then $|T_3^{(w_3)}| = 4$, so this situation satisfies the constraint of $|T_3^{(w_3)}| \geq 4$.
 \item If $p_1 c_6$ were true, then $|T_3^{(w_3)}| = 3$, so this situation does not satisfy the constraint of $|T_3^{(w_3)}| \geq 4$.
 \item Obviously, this also implies that $p_1 c_7$, $p_1 c_8$, $p_1 c_9$ do not satisfy the constraint.   
\end{itemize}
Hence, it is as if it is announced that $[p_1 c_2 \vee p_1 c_5]$:
\begin{equation*}
M'\models [p_1 c_2 \vee p_1 c_5] K_1 (p_1 c_2 \vee p_1 c_5)
\end{equation*}
This means that the probability for player 1 to win can be re-estimated: $\mathcal{P}^{(w_1)}_1 = \frac{1}{2}$.

\subsubsection{Updating player 2's view}
For player 2 we can say that player he knows what the cards of player 1 and 3 are: 
\begin{equation*}
M' \models K_2 p_1c_2 \wedge K_1p_3c_4 
\end{equation*}
Furthermore, he knows:
\begin{equation*}
 M' \models K_2(p_2 c_3 \vee p_2 c_5 \vee p_2 c_6 \vee p_2 c_7 \vee p_2 c_8 \vee p_2 c_9)
\end{equation*}
\begin{itemize}
 \item If $p_2 c_3$ were true, then $|T_3^{(w_3)}| = 6$, so this situation satisfies the constraint of $|T_3^{(w_3)}| \geq 4$.
 \item If $p_2 c_5$ were true, then $|T_3^{(w_3)}| = 4$, so this situation satisfies the constraint of $|T_3^{(w_3)}| \geq 4$.
 \item If $p_2 c_6$ were true, then $|T_3^{(w_3)}| = 3$, so this situation does not satisfy the constraint of $|T_3^{(w_3)}| \geq 4$.
 \item Obviously, this also implies that $p_2 c_7$, $p_2 c_8$, $p_2 c_9$ do not satisfy the constraint.   
\end{itemize}
Hence, it is as if it is announced that $[p_2 c_3 \vee p_1 c_5]$:
\begin{equation*}
M'\models [p_2 c_2 \vee p_2 c_5] K_2 (p_2 c_3 \vee p_2 c_5)
\end{equation*}
This means that the probability for player 2 to win can be re-estimated:$\mathcal{P}^{(w_2)}_2 = \frac{1}{2}$.

This round solely intended to demonstrate the consequence of a player's decision in case their strategy is known.

\subsection{Third round}
In another round we will have the situation in which player 1 receives the card 9, player 2 receives 2 and player 3 receives 6. For simplicity, we will assume that player 3 starts this round as well. It is obvious that $\mathcal{P}^{(w_3)}_3 = 0$. Player 3 feels like luck is on his side and decides to bluff. He know can make use of the fact that other players know his strategy: $M \models K_3 K_1 h_3 \wedge K_3 K_2 h_3$. Let's see whether this bluff is successful or not!
\subsubsection{Updating player 1's view}
For player 1 we have that:
\begin{equation}
 M'\models K_1 p_2 c_2 \wedge K_1 p_3 c_6
\end{equation}
Furthermore, he nows:
\begin{equation*}
M'\models K_1(p_1 c_3 \vee p_1 c_4 \vee p_1 c_5 \vee p_1 c_7 \vee p_1 c_8 \vee p_1 c_9
\end{equation*}
Similar to what happened in the previous round, we have that $|T_3^{(w_3)}| \geq 4$.
\begin{itemize}
 \item If $p_1 c_3$ were true, then $|T_3^{(w_3)}| = 6$, so this situation satisfies the constraint.
 \item If $p_1 c_4$ were true, then $|T_3^{(w_3)}| = 5$, so this situation satisfies the constraint.
 \item If $p_1 c_5$ were true, then $|T_3^{(w_3)}| = 4$, so this situation is also satisfied by the constraint.
 \item If $p_1 c_7$ were true, then $|T_3^{(w_3)}| = 2$, so this situation is not satisfied by the constraint.
 \item Obviously, $p_1 c_8$ and $p_1 c_9$ will not satisfy the constraint either. 
\end{itemize}
Hence, if player 1 believes player 3 to play according to his strategy and utilities, he would falsely assume that he can only hold 3, 4 or 5.

Unfortunately, we cannot express this neatly with public announcement logic, since everything that is announced must be true.

Nevertheless, let us assume that player 1 thinks that only 3, 4 and 5 are valid alternatives for his own card. This means that $\mathcal{P}^{(w_1)}_1 = 0$. Hence, the bluff is successful: player 1 folds.

\subsubsection{Updating player 2's view}
Player 2 has the following knowledge:
\begin{equation*}
 M'\models K_2 p_1 c_9 \wedge K_2 p_3 c_6
\end{equation*}
Moreover, he knows:
\begin{equation*}
M'\models K_2(p_2 c_2 \vee p_2 c_3 \vee p_2 c_4 \vee p_2 c_5 \vee p_2 c_7 \vee p_2 c_8
\end{equation*}

For player 2 it is obvious that player 3 bluffs. Hence, he knows that it is inappropriate to apply the derivations for determining $|T_3^{(w_3)}|$. Therefore, it is not wise to update anything based on his bluffing call. 

Nevertheless, he knows that his gain will be equal to his loss (if he plays there will be just two coins in the pot). Note that $\gamma_2 = 1$. Hence the probability for him to win should be at least $0.5$ to participate. This is not the case, and so he decides to fold as well, leaving no gain nor loss for player 3. 

\subsection{Discussion}
It is interesting to see that from such a simple game situation it is difficult to apply our theory in practice. It is based on a lot of assumptions that may fail to be useful in practice. 


\begin{comment}
\begin{figure}[!h]
 \centering
\begin{tikzpicture}
  \node [mystate] (a) at (0,0) {$\bullet$};
  \node [mystate] (b) at (5,0) {$\bullet$};
  \draw (a) node[below] {$\neg p, q, \neg r$};
  \draw (b) node[below] {$p, q, r$};
  \draw (a) node[above] {$w_1$};
  \draw (b) node[above] {$w_2$};
  \draw (a) to node[below] {$b$} (b);
\end{tikzpicture}
\caption{Simple world}
\label{fig:wrld}
\end{figure}
\end{comment}



\end{document}
